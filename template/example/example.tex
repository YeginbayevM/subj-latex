\documentclass[a4paper,12pt]{article}

% 🔹 Базовые настройки
\usepackage[utf8]{inputenc}
\usepackage[T2A]{fontenc}
\usepackage[russian]{babel}
\usepackage{geometry}
\geometry{top=2cm, bottom=2cm, left=2cm, right=2cm}

% 🔹 Цвет фона и текста
\usepackage{xcolor}
\usepackage{pagecolor}
\definecolor{lightgray}{gray}{0.95}
\definecolor{darkblue}{rgb}{0,0.2,0.6}
\pagecolor{lightgray}
\color{black}

% 🔹 Улучшенное форматирование
\usepackage{titlesec}
\titleformat{\section}{\large\bfseries\color{darkblue}}{}{0pt}{}
\titleformat{\subsection}{\normalsize\bfseries\color{darkblue}}{}{0pt}{}

% 🔹 Колонтитулы
\usepackage{fancyhdr}
\pagestyle{fancy}
\fancyhf{}
\fancyhead[L]{\textbf{Описание дисциплины} — {\color{darkblue} \CourseName}}
\setlength{\headheight}{14.5pt}

% 🔹 Таблицы
\usepackage{longtable, array, booktabs, colortbl, tabularx, makecell, multirow, hhline}
\renewcommand{\arraystretch}{1.}

% 🔹 Красивые блоки
\usepackage{tcolorbox}
\tcbset{colback=white, colframe=darkblue, arc=5pt}

% 🔹 Дополнительное форматирование
\usepackage{ragged2e} % Выравнивание текста
\usepackage{ulem} % Подчеркивание
\usepackage{setspace}
\setstretch{0.9}
\pagenumbering{gobble}

\newcommand{\CourseName}{Пример курса}
\newcommand{\CourseCode}{ABC123}
\newcommand{\Semester}{1}
\newcommand{\Credits}{3}
\newcommand{\CourseDescription}{Краткое описание курса.}
\newcommand{\CourseGoal}{Цель курса — изучение основ.}
\newcommand{\LearningOutcomeOne}{Понимание базовых концепций.}
\newcommand{\LearningOutcomeTwo}{Навыки практического применения.}
\newcommand{\LearningOutcomeThree}{Анализ и решение задач.}

\begin{document}

{\color{darkblue}\section*{\CourseName}}
\begin{tcolorbox}
\textbf{Код курса:} \CourseCode\\
\textbf{Семестр:} \Semester\\
\textbf{Кредитная стоимость:} \Credits
\end{tcolorbox}

\subsection*{Краткое содержание}
\CourseDescription

\subsection*{Цель}
\CourseGoal

\subsection*{Результаты обучения}
\begin{itemize}
    \item \textbf{Навык 1:} \LearningOutcomeOne
    \item \textbf{Навык 2:} \LearningOutcomeTwo
    \item \textbf{Навык 3:} \LearningOutcomeThree
\end{itemize}

\subsection*{Содержание}
\small
\begin{table}[h]
    \centering
    \caption{Содержание курса}
    \renewcommand{\arraystretch}{1.3}
    \arrayrulecolor{black} % Цвет линий (чёрный)
    \begin{tabular}{|c|p{6.5cm}|c|c|c|}
        \hline
        \rowcolor{darkblue!20} 
        \textbf{№} & \textbf{Наименование тем} & \textbf{Лекции} & \textbf{Практики} & \textbf{СРО} \\
        \hline
        1 & Введение в тему & 2 & 4 & 10 \\
        2 & Основные понятия & 1 & 2 & 5 \\
        3 & Основные концепции & 2 & 4 & 12 \\
        4 & Применение на практике & 2 & 4 & 14 \\
        \hline
    \end{tabular}
    \end{table}

\subsection*{Пререквизиты}
Перечень необходимых знаний перед изучением курса.

\subsection*{Литература}
\textbf{Основные книги:}  
1. Название книги 1  
2. Название книги 2  

\textbf{Дополнительные ресурсы:}  
1. Статья/сайт 1  
2. Статья/сайт 2  

\subsection*{Координатор курса}
Имя преподавателя

\subsection*{Использование компьютеров}
Программное обеспечение, используемое на курсе.

\subsection*{Лабораторные работы}
Перечень основных лабораторных работ.

\end{document}
