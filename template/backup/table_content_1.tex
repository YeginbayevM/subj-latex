% Файл table.tex
\begin{table}[H] % Используем [H] для фиксации таблицы
    \centering
    \renewcommand{\arraystretch}{1.1} % Уменьшаем межстрочный интервал
    \arrayrulecolor{darkblue} % Синие линии таблицы
    \rowcolors{2}{darkblue!10}{white} % Чередование цветов строк
    \fontsize{9}{11}\selectfont % Уменьшаем размер шрифта
    \begin{tabular}{|c|p{5.5cm}|c|c|c|}
        \hline
        \rowcolor{darkblue!20} 
        \textbf{№} & \textbf{Наименование тем} & \textbf{Лекции} & \textbf{Практики} & \textbf{СРО} \\
        \hline
        1 & Введение в Computer Science и основы программирования & 1 & 2 & 7 \\
        2 & Основы языка C и работа с интерфейсом командной строки (CLI) & 1 & 2 & 9 \\
        3 & Процесс разработки, массивы, строки, основы криптографии & 2 & 4 & 12 \\
        4 & Основы алгоритмов: поиск, сортировка и рекурсия & 2 & 4 & 10 \\
        5 & Управление памятью: указатели, стек и куча & 2 & 4 & 8 \\
        6 & Структуры данных & 1 & 2 & 8 \\
        7 & Основы языков программирования & 1 & 2 & 7 \\
        8 & Основы SQL & 1 & 2 & 9 \\
        9 & Введение в веб-разработку (HTML, CSS, JS) & 1 & 2 & 9 \\
        10 & Основы операционных систем & 1 & 2 & 7 \\
        11 & Введение в компьютерную графику & 1 & 2 & 7 \\
        12 & Современный искусственный интеллект & 1 & 2 & 8 \\
        \hline
        \rowcolor{darkblue!20} 
        \textbf{Всего} & & \LectureCount & \PracticeCount & \SelfLearnCount \\
        \hline
    \end{tabular}
\end{table}